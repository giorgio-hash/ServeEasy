\section{Toolchain}
Di seguito è presentata la toolchain utilizzata per lo sviluppo del progetto software
\subsection{Modellazione}
\begin{itemize}
	\item draw.io: casi d’uso e topologia;
\end{itemize}

\subsection{Stack applicativo}
\begin{itemize}
	\item Angular.js: front-end;
	\item Java Spring Boot 3.x.x: back-end;
	\item MariaDB: database per l’archiviazione;
	\item H2: database in-memory per rendere più efficiente l’estrazione dei dati;
\end{itemize}

\subsection{Deployment}
\begin{itemize}
	\item Docker: piattaforma per container virtuali;
	\item Docker Compose: gestione app multi-container;
\end{itemize}

\subsection{Gestore repository}
\begin{itemize}
	\item Git: Controllo versione per codice sorgente;
	\item GitHub: Piattaforma hosting e collaborativa per progetti Git;
\end{itemize}

\subsection{Continuous Integration}
\begin{itemize}
	\item Maven: gestore di progetti e dipendenze Java;
	\item GitHub Action: piattaforma di automazione per repository GitHub;
	\item Jenkins: strumento di automazione per sviluppatori;
\end{itemize}

\subsection{Analisi statica}
\begin{itemize}
	\item Checkstyle: visualizzazione di alto livello di metriche qualitative del codice;
\end{itemize}

\subsection{Analisi dinamica}
\begin{itemize}
	\item Postman: strumento per testare API e servizi;
	\item Garfana: analisi delle performance della rete di microservizi;
	\item JUNIT: framework per test unitari Java;
\end{itemize}

\subsection{Documentazione e organizzazione del team}
\begin{itemize}
	\item Google Drive: servizio cloud per archiviazione;
	\item Documenti condivisi di Google: per elaborare la documentazione in modo condiviso;
	\item \LaTeX: generazione documentazione;
	\item Microsoft Teams: per organizzazione e meeting;
\end{itemize}

\subsection{Modello di sviluppo}
Il modello adottato segue la filosofia AGILE, con enfasi sui seguenti aspetti-chiave:
\begin{itemize}
	\item pair programming, per favorire creatività e controllo del lavoro prodotto;
	\item orientamento al risultato, con enfasi maggiore sulla generazione di codice funzionante e componenti completi prima della relativa documentazione;
	\item rapidità di risposta ai cambiamenti;
	\item collaborazione attiva col cliente, al fine di incontrare le sue necessità, garantire trasparenza e fornire feedback tempestivo sul lavoro di progetto;
	\item Proattività nell’identificazione e mitigazione dei rischi.
\end{itemize}

\clearpage