\section{Requisiti non funzionali}
Il progetto verrà sviluppato tenendo considerazione delle performance, integrabilità, modificabilità, testabilità e sicurezza dei componenti.

\subsection{Performance}
L'algoritmo di priorità impiegato dalla cucina per la selezione degli ordini deve fornire risultati in un tempo utile. Allo stesso tempo, gli utenti dell'applicativo web devono poter accedere e aggiornare le informazioni in un tempo accettabile.

\subsection{Integrabilità}
Ogni componente di sistema deve collaborare con gli altri componenti in modo da garantire le funzionalità previste dal sistema. Questa caratteristica è essenziale per garantire il corretto funzionamento e la coerenza dell'intero sistema.

\subsection{Modificabilità}
Il software deve facilitare l’aggiunta di nuovi componenti e funzionalità.

\subsection{Testabilità}
Ogni componente deve poter permettere la progettazione, implementazione ed esecuzione di test efficaci, in modo da garantire una massima copertura di requisiti e funzionalità.

\subsection{Sicurezza}
Il sistema deve integrare meccanismi di autenticazione ed autorizzazione degli attori, in modo da garantire la gestione delle identità, oltre alla protezione dei dati e delle API da accessi non autorizzati. Risulta dunque necessaria una distinzione dei ruoli con cui gli attori accedono al sistema.

\clearpage