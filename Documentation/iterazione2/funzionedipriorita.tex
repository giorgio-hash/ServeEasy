\section{Funzione di priorità}
La funzione di priorità è una funzione matematica che assegna un valore numerico decimale di priorità nell’intervallo tra 0 e 1 basandosi sui parametri specifici di ogni ordine.
Il primo passo consiste nel processo di normalizzazione dei parametri, il quale permette di standardizzare i valori in modo che siano compresi tra 0 e 1. in maniera tale da mettere i diversi parametri su una scala comune e uniforme

\subsection{Parametri}

\paragraph{x1 ingrediente principale:}
Indica il valore di priorità che presenta l'ingrediente predominante dell’ordine, questo valore è influenzato direttamente dallo stato della postazione di lavoro associata in cucina attraverso il grado di riempimento della coda di postazione.
\begin{itemize}
	\item se la cucina è satura ridurre il valore di x1 (min 0)
	\item se la cucina è scarica aumentare il valore di x1(max 1)
\end{itemize}

\paragraph{x2 tempo di preparazione:}
Rappresenta la durata stimata necessaria per preparare un determinato ordine.
\begin{itemize}
	\item normalizzazione: 
	\begin{equation*}
		\text{tp}_{\text{norm}} = \frac{\text{tp} - \text{tp}_{\text{min}}}{\text{tp}_{\text{max}} - \text{tp}_{\text{min}}}	
	\end{equation*}
	
	\item \text{tp}: tempo di preparazione
	\item $\text{tp}_{\text{max}}$: tempo di preparazione massimo;
	\item $\text{tp}_{\text{min}}$: tempo di preparazione minimo;
	\item Considerare $x2 = \text{tp}_{\text{norm}}$ per prioritizzare ordini più lunghi, oppure $x2 = 1-\text{tp}_{\text{norm}}$ per prioritizzare ordini più brevi.
\end{itemize}

\paragraph{x3 urgenza del cliente:}
Consente ai clienti di specificare la tempestività con cui desiderano ricevere il proprio ordine, in particolare i clienti possono chiedere espressamente di avere urgenza, al contrario possono non dire nulla e tenere un valore di urgenza di default.
\begin{itemize}
	\item 1 se il cliente ha espresso urgenza;
	\item 0 se il cliente non ha espresso alcuna urgenza - \textit{parametro di default}.
\end{itemize}

\paragraph{x4 numero ordine effettuato:}
Specifica il numero dell’ordine del cliente in ordine temporale, in particolare indica la posizione relativa di un ordine all'interno della sequenza di ordini effettuati da egli stesso.
\begin{itemize}

	\item il primo ordine effettuato ha priorità maggiore, mentre i successivi hanno priorità decrescente data dalla normalizzazione, fino ad un numero di ordine soglia oltre il quale la priorità sarà minima;
	
	\item funzione: \[ f(\text{noe}) = \begin{cases} 1 - \frac{{\text{noe} - 1}}{{\text{sogliaMax} - 1}} & \text{se } \text{noe} \leq \text{sogliaMax} \\ 0 & \text{se } \text{noe} > \text{sogliaMax} \end{cases} \]

	\item \textbf{noe} \text{: numero ordine effettuato;}
	\item \textbf{sogliaMax} \text{: massimo numero ordini effettuabili;}
	\item \textbf{ESEMPIO: } \textit{sia dato sogliaMax = 5, il 1° avrà priorità 1 (massima), dal 2° al 5° avremo priorità normata, dopo 5 ordine il parametro x4 avrà priorità 0 (minima).}
\end{itemize}

\paragraph{x5 tempo in attesa:}
Rappresenta il periodo di tempo trascorso da quando un ordine è stato effettuato fino al momento in cui viene elaborato.
\begin{itemize}
	\item funzione: \[ f(\text{tempo di attesa}) = \begin{cases} \frac{\text{tempo di attesa}}{\text{tempo max in attesa}} & \text{se } \text{tempo di attesa} \leq \text{tempo max di attesa} \\ 1 & \text{se } \text{tempo di attesa} > \text{tempo max di attesa} \end{cases} \]

	\item \textbf{tempo di attesa:} è dato dalla differenza fra istante di tempo attuale e istante di tempo della richiesta ordine;
	\item \textbf{tempo max di attesa:} massimo tempo di attesa ordini oltre il quale la priorità è massima;
	\item considerare un valore massimo di tempo in attesa consentito, in prossimità del quale si ha la priorità più elevata.
\end{itemize}

\subsection{Pesi}
I pesi sono utilizzati per attribuire un grado di importanza relativo a ciascun parametro all'interno della funzione di priorità. Questi pesi indicano quanto ciascun parametro dovrebbe influenzare il calcolo complessivo della priorità di un determinato elemento.
Si elencano di seguito i pesi per ciascun parametro definito poc’anzi:
\begin{itemize}
	\item p1: peso ingrediente principale;
	\item p2: peso tempo di preparazione;
	\item p3: peso urgenza cliente;
	\item p4: peso numero ordine;
	\item p5: peso tempo in attesa.
\end{itemize}

Viene quindi fatto un ragionamento sull’importanza da attribuire a ogni parametro tramite l’incidenza assegnata al singolo peso. Si dividono quindi i pesi in tre categorie.

\paragraph{Maggiore incidenza} I pesi che devono essere più incidenti sono:
\begin{itemize}
	\item p1 peso ingrediente principale: per evitare di sovraccaricare una postazione rispetto alle altre o per non avere postazioni vuote;
	\item p5 peso tempo in attesa: un ordine non può restare in attesa troppo a lungo.
\end{itemize}

\paragraph{Incidenza media} Il peso con incidenza media è:
\begin{itemize}
	\item p3 peso urgenza del cliente: è meno importante dei vincoli di sovraccarico e attesa, ma deve essere comunque una scelta significativa.
\end{itemize}

\paragraph{Bassa incidenza} I pesi con bassa incidenza sulla priorità sono:
\begin{itemize}
	\item p2 peso tempo di preparazione: in confronto ad altri parametri con pesi più elevati, questo è considerato meno critico;
	\item p4 peso numero ordine effettuato: ha un impatto di poco conto sulla priorità dell’ordine.
\end{itemize}

\subsubsection{Valore dei pesi}
I valori dei pesi vengono quindi definiti inizialmente:
\begin{itemize}
	\item p1 = 0.25;
	\item p2 = 0.15;
	\item p3 = 0.20;
	\item p4 = 0.15;
	\item p5 = 0.25.
\end{itemize}
\paragraph{Prospettive Future} Questi valori possono essere regolati col tempo per aumentare l’efficienza dell’algoritmo, diventa così importante raccogliere dati storici per poterli analizzare e comprendere come i vari parametri influenzano le prestazioni del sistema, oltre a raccogliere feedback dei clienti, sulla base di ciò sarà richiesto un tuning dei pesi più accurato, ad esempio tramite un modello di machine learning. Per questo motivo è richiesta una certa flessibilità in modo da consentire l'aggiornamento dei pesi dei parametri in modo dinamico.

\subsection{Funzione matematica}
L'equazione proposta rappresenta una somma pesata dei parametri, dove ciascun parametro (x1, x2, x3, x4, x5) viene moltiplicato per il suo relativo peso (p1, p2, p3, p4, p5). I pesi indicano l'importanza relativa dei parametri nel determinare la priorità complessiva di un elemento. La somma pesata dei parametri produce un valore (y) compreso tra 0 e 1, dove 0 indica un valore meno urgente e 1 indica un valore più urgente.
\begin{equation*}
	\text{y} = \text{p1}*{\text{x1}} + \text{p2}*{\text{x2}}  + \text{p3}*{\text{x3}}  + \text{p4}*{\text{x4}}  + \text{p5}*{\text{x5}}
\end{equation*}

\clearpage